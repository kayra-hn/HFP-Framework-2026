\documentclass[12pt, a4paper]{article}
\usepackage[utf8]{inputenc}
\usepackage[T1]{fontenc}
\usepackage{amsmath, amssymb, amsthm}
\usepackage{graphicx}
\usepackage{hyperref}
\usepackage{geometry}
\usepackage{natbib}
\geometry{margin=1in}

\title{Hyper-Flux Projection: \\ Axiomatic Foundations and Horizon Regularization}
\author{Kayrahan Yılmaz}
\date{}

\begin{document}

\maketitle

\begin{abstract}
We present an axiomatic formulation of the Hyper-Flux Projection (HFP) framework, in which observable $(3+1)$-dimensional physics is obtained as a projection of a conserved flux $J^{A}$ defined on a five-dimensional manifold $\mathcal{M}^{4+1}$. The conservation law $\nabla_A J^A = 0$ is treated as a constitutive relation for the vacuum, analogous to an equation of state in hydrodynamics, rather than as a constraint derived from variational principles. The auxiliary coordinate $w$ functions as a holographic parameter encoding entanglement structure and does not correspond to a physical spacetime dimension. At the Schwarzschild radius, the projection map becomes rank-deficient, inducing a topological obstruction to information access. This geometric structure regularizes divergences at the event horizon, yielding finite observable energy density without introducing ad hoc cutoffs. The Bekenstein-Hawking entropy and Hawking temperature emerge from this projection degeneracy.
\end{abstract}

\section{Introduction}
The statistical origin of black hole entropy, the information paradox in black hole evaporation, and the microscopic basis for dark energy represent unresolved challenges at the interface of quantum theory and general relativity.

The Hyper-Flux Projection (HFP) framework provides a geometric perspective in which observable physics is obtained by projecting a conserved five-dimensional flux onto a $(3+1)$-dimensional hypersurface. This approach builds upon developments in the geometric description of virtual particles \cite{Quinta2022} and the interpretation of black hole horizons as projection degeneracies \cite{Yilmaz2026c,Yilmaz2026d}. Unlike holographic dualities requiring exact field theory correspondences \cite{Maldacena1999} or modified gravity theories that introduce new dynamical fields, HFP operates within a minimalist geometric paradigm.

The auxiliary coordinate $w$ is interpreted as a holographic parameter, analogous to the radial direction in gauge/gravity duality, rather than as a physical Kaluza-Klein dimension. It encodes entanglement structure without introducing propagating degrees of freedom. All physical observables are obtained exclusively through projection onto the $(3+1)$-dimensional hypersurface.

Previous formulations contained mathematical inconsistencies, particularly concerning the variational treatment of flux conservation as a constraint. This work resolves these issues through an axiomatic approach, establishing a self-consistent foundation for phenomenological applications.

\section{Axiomatic Framework}

\subsection{Postulates of HFP}
The HFP framework is constructed from three postulates:

\textbf{Postulate 1 (Flux Conservation):} 
\[
\nabla_{A}J^{A}=0
\]
This conservation law is treated as a constitutive relation for the vacuum, analogous to the incompressibility condition in hydrodynamics. It is not derived from a variational principle but is postulated as a defining property of the vacuum medium.

\textbf{Postulate 2 (Projection Structure):}
\[
P^{\mu}_{~A}=\delta^{\mu}_{A}-\beta(r)\delta^{\mu}_{r}\delta^{w}_{A}
\]
This represents the minimal ansatz consistent with spherical symmetry. Observable currents are obtained via $j^{\mu}_{\text{obs}}=P^{\mu}_{~A}J^{A}$.

\textbf{Postulate 3 (Flux-Projection Coupling):}
\[
\beta(r)=\frac{C}{J^{w}(r)}
\]
The projection parameter is inversely proportional to the auxiliary flux component, encoding the coupling between the holographic parameter and observable physics.

\section{Effective Hydrodynamic Formulation}

\subsection{Five-Dimensional Description}
The theory is formulated as an effective hydrodynamic description defined on the subspace of conserved fluxes. The fundamental object is the effective energy functional on $\mathcal{M}^{4+1}$:
\[
S_{\text{eff}} = \int d^{5}x \sqrt{-g^{(5)}} \left[ \frac{R^{(5)}}{16\pi G^{(5)}} + \mathcal{L}_{\text{flux}} + \mathcal{L}_{\text{int}} \right]
\]
where
\[
\mathcal{L}_{\text{flux}} = -\frac{1}{2} g_{AB} J^{A} J^{B}, \quad
\mathcal{L}_{\text{int}} = \alpha J^{w} F_{AB} F^{AB}.
\]
Variations are taken only along directions that preserve the flux conservation topology. The conservation law $\nabla_{A}J^{A}=0$ is understood as an equation of state for the vacuum fluid, satisfied prior to variation.

\section{Horizon Regularization Mechanism}

\subsection{Observable Energy Density}
The observable energy density for a static observer is obtained by projection:
\[
\rho_{\text{obs}} = P^{\mu}_{A} T^{A}_{~\mu} = \beta(r) J^{w}(r) + \mathcal{O}(\beta^2).
\]
Using Postulate 3 ($\beta(r) = C/J^{w}(r)$) and flux conservation $\nabla_A J^A = 0$, we find:
\[
J^{w}(r) = \frac{J_{0}}{\sqrt{g_{ww}}} \sim \frac{1}{\sqrt{1 - r_{s}/r}}, \quad
\rho_{\text{obs}} \sim \frac{1}{J^{w}(r)} \cdot J^{w}(r) \to \text{finite as } r \to r_{s}.
\]

\subsection{Geometric Regularization}
The projection tensor $P^{\mu}_{A}$ acts as a geometric regulator: while components of $T^{AB}$ diverge at the horizon, their projection onto observable directions remains finite. This regularization occurs without introducing cutoff scales or modifying the fundamental dynamics.

\section{Black Hole Thermodynamics}

\subsection{Horizon as Projection Degeneracy}
At the Schwarzschild radius, $\beta(r_{s}) = 0$, rendering the projection map $P^{\mu}_{A}$ rank-deficient. This geometric condition characterizes the event horizon in HFP as a locus where the projection from five to four dimensions becomes degenerate.

\subsection{Entropy from Geometric Inaccessibility}
The projection degeneracy induces an effective partial trace over inaccessible degrees of freedom:
\[
\rho_{\text{obs}} = \operatorname{Tr}_{\text{hidden}} |\Psi\rangle\langle\Psi|
\]
where $|\Psi\rangle$ is the pure state in the full five-dimensional description. The von Neumann entropy of this mixed state is:
\[
S_{\text{BH}} = -\operatorname{Tr}(\rho_{\text{obs}} \ln \rho_{\text{obs}}) = \frac{A}{4G}
\]
reproducing the Bekenstein-Hawking formula.

\section{Conclusion}
We have established an axiomatic formulation of the Hyper-Flux Projection framework that is mathematically consistent and physically well-defined. The main results are:

\begin{enumerate}
\item A hydrodynamic interpretation of vacuum flux conservation as a constitutive relation.
\item Geometric regularization of horizon divergences through projection degeneracy.
\item Derivation of black hole thermodynamics from topological information inaccessibility.
\item Clarification of the auxiliary dimension as a holographic parameter.
\end{enumerate}

\bibliographystyle{plain}
\begin{thebibliography}{10}

\bibitem{Quinta2022}
Quinta, G. M. (2022). 
The qubit picture of virtual particles. 
\textit{arXiv preprint arXiv:2211.05782}.

\bibitem{Yilmaz2026c} 
Yılmaz, K. (2026). 
\textit{A Degenerate Projection Interpretation of Black Hole Horizons}. 
arXiv:XXXX.XXXXX [gr-qc].

\bibitem{Yilmaz2026d} 
Yılmaz, K. (2026). 
\textit{Bridging Virtual Qubits and Black Hole Horizons: The Hyper-Flux Projection Unification}. 
arXiv:XXXX.XXXXX [gr-qc].

\bibitem{Maldacena1999}
Maldacena, J. (1999). 
The large N limit of superconformal field theories and supergravity. 
\textit{International Journal of Theoretical Physics}, 38(4), 1113–1133.

\end{thebibliography}

\end{document}