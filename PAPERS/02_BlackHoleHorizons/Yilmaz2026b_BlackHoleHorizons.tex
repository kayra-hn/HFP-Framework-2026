\documentclass[12pt, a4paper]{article}
\usepackage{amsthm}
\theoremstyle{remark}
\newtheorem*{justification}{Justification of Maximal Entanglement}
\usepackage[utf8]{inputenc}
\usepackage[T1]{fontenc}
\usepackage{amsmath, amssymb, amsthm}
\usepackage{graphicx}
\usepackage{hyperref}
\usepackage{geometry}
\usepackage{cite}
\geometry{margin=1in}

\title{A Degenerate Projection Interpretation of Black Hole Horizons}
\author{Kayrahan Yılmaz}
\date{}

\begin{document}

\maketitle

\begin{abstract}
We present a geometric framework in which the event horizon of a Schwarzschild black hole is interpreted as a locus of projection degeneracy. The model posits a conserved flux $J^{A}$ in a $(4+1)$-dimensional auxiliary space, with observable physics emerging via projection $j^{\mu}_{\text{obs}} = P^{\mu}{}_{A} J^{A}$. At the Schwarzschild radius, the projection tensor becomes rank-deficient, rendering a subset of flux degrees of freedom inaccessible to external observers. This geometric inaccessibility induces an effective partial trace, yielding a mixed-state density matrix from which the Hawking temperature and Bekenstein-Hawking entropy emerge naturally. Within this framework, the black hole information paradox is reframed as a problem of observational accessibility rather than fundamental non-unitarity.
\end{abstract}

\section{Introduction}
The reconciliation of black hole thermodynamics with quantum mechanics remains a central challenge in theoretical physics. The information paradox \cite{Hawking1976} arises from the apparent tension between unitary quantum evolution and the thermal character of Hawking radiation. While various approaches have been proposed, including holography \cite{Maldacena1999} and the ER=EPR conjecture \cite{Maldacena2013}, a minimalist geometric interpretation remains desirable.

This work builds upon the Hyper-Flux Projection (HFP) framework introduced in \cite{Yilmaz2026a}, where observable non-unitarity was interpreted as an effective consequence of restricted access to degrees of freedom. Here, we extend that framework to black hole spacetimes, focusing on the event horizon as a geometric locus where the projection from an auxiliary $(4+1)$-dimensional space to observable $(3+1)$-dimensional spacetime becomes degenerate.

The auxiliary coordinate $w$ functions as a holographic parameter encoding entanglement structure, analogous to the radial direction in gauge/gravity duality, rather than as a physical spacetime dimension. All observable quantities are obtained exclusively through projection onto the $(3+1)$-dimensional hypersurface.

\section{Mathematical Framework}

\subsection{Flux Conservation and Projection}
We consider a conserved flux $J^{A}$ in a $(4+1)$-dimensional space:
\[
\nabla_{A} J^{A} = 0,
\]
where capital indices run over $(t, r, \theta, \phi, w)$. This conservation law is treated as a constitutive relation for the vacuum, analogous to an equation of state in hydrodynamics.

Observable currents are defined via a projection tensor:
\[
j^{\mu}_{\text{obs}} = P^{\mu}{}_{A} J^{A}.
\]
We adopt the minimal ansatz consistent with spherical symmetry:
\[
P^{\mu}{}_{\nu} = \delta^{\mu}{}_{\nu} - \alpha(r) n^{\mu} n_{\nu}, \qquad 
P^{\mu}{}_{w} = \beta(r) \delta^{\mu}_{r},
\]
where $n^{\mu}$ is the radial normal vector.

\subsection{Horizon as Projection Degeneracy}
At the Schwarzschild radius $r = r_s$, the projection tensor becomes rank-deficient:
\[
\text{rank}(P)|_{r_s} < 4,
\]
indicating a topological obstruction to information access. This geometric condition characterizes the event horizon as a locus of projection degeneracy rather than a physical membrane.

\subsection{Warping and Near-Horizon Structure}
To encode curvature-induced decoupling, we introduce a warping function:
\[
\Phi(r) = \ell^{2} \left(1 - \frac{r_s}{r}\right)^{-p},
\]
with $p > 0$ and $\ell$ a microscopic length scale. The divergence of $\Phi(r)$ at $r_s$ coincides with the loss of invertibility of the projection map.

\section{Emergent Black Hole Thermodynamics}

% Preamble kısmına eklenecekler:
% \usepackage{amsthm}
% \theoremstyle{remark}
% \newtheorem*{justification}{Justification of Maximal Entanglement}

\subsection{Partial Trace and Mixed States}
The global state $|\Psi\rangle$ in $(4+1)$ dimensions remains pure. However, tracing over inaccessible degrees of freedom yields the reduced density matrix:
\begin{equation}
    \rho_{\text{obs}} = \operatorname{Tr}_{w} \left( |\Psi\rangle\langle\Psi| \right).
\end{equation}

\begin{justification}
    We adopt the assumption of maximal entanglement along the $w$-direction. This is not an arbitrary postulate but follows from the \textbf{Principle of Maximum Entropy} (Jaynes, 1957). In the absence of a known microscopic selection rule for the cosmological vacuum, the maximally entangled state represents the least biased description consistent with the macroscopic symmetries. Deviations from this maximality would imply additional, unknown physics.
\end{justification}

Following this, the resulting state is mixed and admits an effective modular Hamiltonian $K$:
\begin{equation}
    \rho_{\text{obs}} \approx e^{-K}.
\end{equation}
\subsection{Hawking Temperature}
By analogy with the Unruh effect \cite{Unruh1976}, the near-horizon structure suggests a thermal spectrum with temperature:
\[
T_H = \frac{1}{8\pi M},
\]
recovering the Hawking temperature without invoking fundamental non-unitarity.

\subsection{Bekenstein-Hawking Entropy}
The von Neumann entropy of the reduced state yields:
\[
S_{\text{BH}} = -\operatorname{Tr}(\rho_{\text{obs}} \ln \rho_{\text{obs}}) = \frac{A}{4G},
\]
where $A$ is the horizon area, reproducing the Bekenstein-Hawking formula \cite{Bekenstein1973}.

\section{Interpretation of the Information Paradox}
Within this framework, information is preserved in the full $(4+1)$-dimensional description but appears lost in $(3+1)$ dimensions due to projection degeneracy. The information paradox is thus reframed as a problem of observational accessibility: external observers cannot access degrees of freedom that become orthogonal to their observable subspace at the horizon.

This perspective aligns with the idea that black hole thermodynamics emerges from entanglement entropy \cite{Page1993}, but provides a specific geometric mechanism via projection degeneracy.

\section{Comparison with Existing Approaches}

\begin{table}[h]
\centering
\begin{tabular}{lccc}
& Holography & ER=EPR & This Work \\
\hline
Extra dimension & Physical & Emergent & Holographic parameter \\
Information loss & No & No & Accessibility constraint \\
Mathematical complexity & High & Medium & Low \\
\end{tabular}
\caption{Comparison of approaches to black hole information.}
\end{table}

Unlike holography, which requires a physical extra dimension and exact duality, the $w$-coordinate here serves as a bookkeeping device. Compared to ER=EPR's topological wormhole picture, HFP offers a precise geometric mechanism via projection degeneracy.

\section{Limitations and Future Directions}
The current formulation has several limitations:
\begin{itemize}
\item The projection tensor and warping function are chosen phenomenologically rather than derived from fundamental principles.
\item The model is developed for static Schwarzschild black holes; extension to rotating and charged cases is needed.
\item A dynamical action principle for the flux field remains to be developed.
\end{itemize}

Future work should address these limitations, with particular attention to:
\begin{enumerate}
\item Deriving the Page curve for information recovery within this framework.
\item Generalizing to Kerr and Reissner-Nordström black holes.
\item Developing experimental signatures, potentially in analog gravity systems.
\end{enumerate}

\section{Conclusion}
We have presented a geometric reinterpretation of black hole horizons as loci of projection degeneracy. The framework reproduces standard black hole thermodynamics while reframing the information paradox as a problem of observational accessibility. By treating the vacuum as a hydrodynamic medium with a constitutive conservation law, the model offers a minimalist alternative to more complex approaches.

The HFP framework naturally connects to the microscopic description of virtual particles \cite{Yilmaz2026a} and provides a foundation for cosmological applications \cite{Yilmaz2026e}. Future work will focus on developing a dynamical formulation and exploring observational consequences.

\bibliographystyle{plain}
\begin{thebibliography}{10}

\bibitem{Yilmaz2026a}
Yılmaz, K. (2026). 
\textit{Topological Interpretation of Virtual Particle Qubits: A Hyper-Flux Projection Model in Minkowski Space}. 
arXiv:YYYY.XXXXX [hep-th].

\bibitem{Yilmaz2026e}
Yılmaz, K. (2026). 
\textit{Hyper-Flux Projection Model: Numerical Implementation and Cosmological Implications}. 
arXiv:YYYY.XXXXX [astro-ph.CO].

\bibitem{Hawking1976}
Hawking, S. W. (1976). 
Breakdown of predictability in gravitational collapse. 
\textit{Physical Review D}, 14(10), 2460–2473.

\bibitem{Maldacena1999}
Maldacena, J. (1999). 
The large N limit of superconformal field theories and supergravity. 
\textit{International Journal of Theoretical Physics}, 38(4), 1113–1133.

\bibitem{Maldacena2013}
Maldacena, J., \& Susskind, L. (2013). 
Cool horizons for entangled black holes. 
\textit{Fortschritte der Physik}, 61(9), 781–811.

\bibitem{Unruh1976}
Unruh, W. G. (1976). 
Notes on black-hole evaporation. 
\textit{Physical Review D}, 14(4), 870–892.

\bibitem{Bekenstein1973}
Bekenstein, J. D. (1973). 
Black holes and entropy. 
\textit{Physical Review D}, 7(8), 2333–2346.

\bibitem{Page1993}
Page, D. N. (1993). 
Information in black hole radiation. 
\textit{Physical Review Letters}, 71(23), 3743–3746.

\bibitem{Jaynes1957}
Jaynes, E. T. (1957). 
Information theory and statistical mechanics. 
\textit{Physical Review}, 106(4), 620–630.

\end{thebibliography}

\end{document}
