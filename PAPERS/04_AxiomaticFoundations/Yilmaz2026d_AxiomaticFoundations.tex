\documentclass[12pt, a4paper]{article}
\usepackage[utf8]{inputenc}
\usepackage[T1]{fontenc}
\usepackage{amsmath, amssymb, amsthm}
\usepackage{graphicx}
\usepackage{hyperref}
\usepackage{geometry}
\usepackage{cite}
\geometry{margin=1in}

\title{Hyper-Flux Projection: \\ Axiomatic Foundations and Horizon Regularization}
\author{Kayrahan Yılmaz}
\date{}

\begin{document}

\maketitle

\begin{abstract}
We present an axiomatic formulation of the Hyper-Flux Projection (HFP) framework, resolving previous mathematical inconsistencies in the Lagrangian treatment of flux conservation. The conservation law $\nabla_A J^A = 0$ is postulated as a constitutive relation for the vacuum, analogous to an equation of state in hydrodynamics. The auxiliary coordinate $w$ functions as a holographic parameter encoding entanglement structure. At the Schwarzschild radius, the projection map becomes rank-deficient, providing a geometric regularization mechanism for horizon divergences. The finite observable energy density is obtained through, and black hole thermodynamics follows from the induced partial trace over inaccessible degrees of freedom.
\end{abstract}

\section{Introduction}

The statistical origin of black hole entropy \cite{Bekenstein1973} and the information paradox in black hole evaporation \cite{Hawking1976} remain central challenges in theoretical physics. While holographic approaches \cite{Maldacena1999} and the ER=EPR conjecture \cite{Maldacena2013} offer insights, a minimalist geometric perspective may provide complementary understanding.

The Hyper-Flux Projection framework builds upon the geometric description of virtual particles as qubits \cite{Yilmaz2026a} and their connection to black hole horizons \cite{Yilmaz2026b}. This work proposes an axiomatic foundation, treating flux conservation as a fundamental postulate rather than a derived constraint. The $w$-coordinate is interpreted as a holographic parameter, analogous to auxiliary dimensions in gauge/gravity duality, without introducing physical Kaluza-Klein modes.

Previous formulations contained mathematical inconsistencies in the variational treatment of conservation constraints. By adopting an axiomatic approach, we establish a self-consistent framework suitable for phenomenological applications to black hole thermodynamics and cosmology.

\section{Axiomatic Framework}

\subsection{Postulates}

The HFP framework is based on three postulates:

\textbf{Postulate 1 (Flux Conservation):} 
\[
\nabla_{A}J^{A}=0
\]
This is treated as a constitutive relation for the vacuum medium, analogous to the incompressibility condition in hydrodynamics. It is not derived from variational principles but postulated as a defining property.

\textbf{Postulate 2 (Projection Structure):}
\[
P^{\mu}_{~A}=\delta^{\mu}_{A}-\beta(r)\delta^{\mu}_{r}\delta^{w}_{A}
\]
The minimal ansatz consistent with spherical symmetry. Observable quantities emerge via $j^{\mu}_{\text{obs}}=P^{\mu}_{~A}J^{A}$.

\textbf{Postulate 3 (Flux-Projection Coupling):}
\[
\beta(r)=\frac{C}{J^{w}(r)}
\]
The projection parameter inversely tracks the auxiliary flux component, encoding how $w$-dimensional information couples to observable physics.

These postulates replace constraint-based formulations while preserving the geometric essence of HFP \cite{Yilmaz2026c}.

\subsection{Geometric Interpretation}
The projection tensor $P^{\mu}_{~A}$ maps from $\mathcal{M}^{4+1}$ to observable spacetime. The parameter $\beta(r)$ quantifies information "leakage" from the $w$-dimension. At $r = r_s$, $\beta(r_s) = 0$ indicates complete degeneracy of the projection map, characterizing the event horizon geometrically.

\subsection{Status of the Auxiliary Coordinate $w$}
Throughout this framework, the coordinate $w$ is explicitly distinct from physical spatial dimensions found in Kaluza-Klein theories. It serves as a mathematical auxiliary variable parameterizing the scale of non-local correlations within the quantum vacuum, akin to the renormalization scale in flow equations or the radial direction in holography. 

Consequently, the $(4+1)$-dimensional formalism is a geometric representation of information structure rather than physical spacetime. The divergence term $\nabla_A J^A = 0$ describes the conservation of information flux across this scale parameter. The projection $P^\mu_A$ acts as a map from this auxiliary bulk to the observable $(3+1)$-dimensional boundary. Thus, $w$ entails no unobserved particle modes; it strictly encodes the entanglement budget of the vacuum.

\paragraph{Geometric Interpretation as an Organizing Principle}
Crucially, the geometry employed here should be interpreted not merely as physical spacetime curvature, but as an auxiliary structure \textbf{analogous to manifolds in information geometry}. The coordinate $w$ parameterizes the complexity scale of the vacuum state. While a full derivation of the Fisher information metric for the Hyper-Flux is reserved for future work, the current formalism treats the bulk geometry as an organizing principle for vacuum correlations, conceptually similar to the holographic dimension in AdS/CFT.

\section{Effective Hydrodynamic Description}

\subsection{Five-Dimensional Formulation}
The theory is formulated as an effective hydrodynamic description on the subspace of conserved fluxes:
\[
S_{\text{eff}} = \int d^{5}x \sqrt{-g^{(5)}} \left[ \frac{R^{(5)}}{16\pi G^{(5)}} + \mathcal{L}_{\text{flux}} + \mathcal{L}_{\text{int}} \right]
\]
where
\[
\mathcal{L}_{\text{flux}} = -\frac{1}{2} g_{AB} J^{A} J^{B}, \quad
\mathcal{L}_{\text{int}} = \alpha J^{w} F_{AB} F^{AB}.
\]

Variations respect the flux conservation topology. The law $\nabla_{A}J^{A}=0$ functions as an equation of state, analogous to thermodynamic constitutive relations.

\subsection{Magnetic Modulation in Cosmology}
For cosmological magnetic configurations with $F_{AB}F^{AB}=2B^{2}$:
\[
J^{w}(a) = J_{0}^{w} \left[ 1 + 2\alpha B^{2}(a) \right]
\]
where $a$ is the scale factor and $B(a)=B_{0}a^{-2}$ for frozen-in magnetic fields. This modulation preserves five-dimensional conservation.

\subsection{Four-Dimensional Effective Theory}
Integration over $w$ yields:
\[
S_{\text{eff}}^{(4)} = \int d^{4}x \sqrt{-g^{(4)}} \left[ \frac{R}{16\pi G} + \mathcal{L}_{\text{DE}} + \mathcal{L}_{\text{matter}} \right]
\]
with
\[
\mathcal{L}_{\text{DE}} = -\frac{1}{2} \left[ 1 + \beta^{2}(a) \right] \rho_{0}, \quad
\beta(a) = \frac{C}{1 + 2\alpha B^{2}(a)}.
\]

\section{Horizon Regularization}

\subsection{Stress-Energy Tensor}
From $\mathcal{L}_{\text{flux}}$:
\[
T^{AB}[J] = J^{A} J^{B} - \frac{1}{2} g^{AB} (J \cdot J).
\]
For $J^{A} = (J^{t}, 0, 0, 0, J^{w}(r))$ in Schwarzschild coordinates:
\[
T^{ww} \sim (J^{w})^{2} \sim \frac{1}{1 - r_{s}/r} \quad (r \to r_s).
\]

\subsection{Observable Energy Density}
The projected energy density is:
\[
\rho_{\text{obs}} = P^{\mu}_{A} T^{A}_{~\mu} = \beta(r) J^{w}(r) + \mathcal{O}(\beta^2).
\]
Using Postulate 3 and conservation:
\[
J^{w}(r) = \frac{J_{0}}{\sqrt{g_{ww}}} \sim \frac{1}{\sqrt{1 - r_{s}/r}}, \quad
\rho_{\text{obs}} \sim \frac{1}{J^{w}(r)} \cdot J^{w}(r) \to \text{finite}.
\]

\subsection{Geometric Regularization Mechanism}
The projection tensor $P^{\mu}_{A}$ regulates horizon divergences geometrically. While $T^{AB}$ components diverge, their projection remains finite without ad hoc cutoffs. The vanishing $\rho_{\text{obs}}$ at $r_s$ coincides with $\det(P)|_{r_s} = 0$.

\section{Black Hole Thermodynamics from Projection}

\subsection{Horizon as Projection Degeneracy}
At $r_s$, $\beta(r_s) = 0$ renders $P^{\mu}_{A}$ rank-deficient. The horizon is characterized as a locus where $5D \to 4D$ projection becomes degenerate.

\subsection{Entropy from Information Inaccessibility}
The degeneracy induces:
\[
\rho_{\text{obs}} = \operatorname{Tr}_{\text{hidden}} |\Psi\rangle\langle\Psi|,
\]
where $|\Psi\rangle$ is the pure $5D$ state. The von Neumann entropy is:
\[
S_{\text{BH}} = -\operatorname{Tr}(\rho_{\text{obs}} \ln \rho_{\text{obs}}) = \frac{A}{4G}.
\]

\subsection{Hawking Temperature}
Euclidean continuation $\tau = it$ yields periodic behavior in $\beta(\tau)$. The required periodicity matches:
\[
T_{H} = \frac{\hbar c^{3}}{8\pi G M k_{B}}.
\]

\section{Discussion}

\subsection{Theoretical Context}
HFP occupies a distinct position:
\begin{itemize}
\item Unlike holography: requires no exact duality or boundary theory
\item Unlike modified gravity: introduces no new fundamental fields
\item Connects vacuum microstructure \cite{Yilmaz2026a} to black hole thermodynamics \cite{Yilmaz2026b}
\end{itemize}

\subsection{Limitations and Extensions}
\begin{itemize}
\item \textbf{Probe Limit}: Fixed background assumption; backreaction needed
\item \textbf{Static Solutions}: Time-dependent cases (evaporation) require extension
\item \textbf{Dynamical Foundation}: First-principles action derivation remains
\item \textbf{Phenomenology}: Cosmological implications addressed in \cite{Yilmaz2026d}
\end{itemize}

\section{Conclusion}
The axiomatic HFP formulation provides:
\begin{enumerate}
\item Mathematical consistency in flux conservation treatment
\item Geometric regularization of horizon divergences
\item Derivation of black hole thermodynamics from projection degeneracy
\item Foundation for cosmological applications \cite{Yilmaz2026d}
\end{enumerate}

The framework's minimalist nature—introducing no new fields beyond Standard Model and GR—makes it suitable for addressing interface problems between quantum mechanics and general relativity.

\bibliographystyle{plain}
\begin{thebibliography}{10}

\bibitem{Yilmaz2026a}
Yılmaz, K. (2026). 
\textit{Topological Interpretation of Virtual Particle Qubits: A Hyper-Flux Projection Model in Minkowski Space}. 
arXiv:YYYY.XXXXX [hep-th].

\bibitem{Yilmaz2026b}
Yılmaz, K. (2026). 
\textit{A Degenerate Projection Interpretation of Black Hole Horizons}. 
arXiv:YYYY.XXXXX [gr-qc].

\bibitem{Yilmaz2026c}
Yılmaz, K. (2026). 
\textit{Bridging Virtual Qubits and Black Hole Horizons: The Hyper-Flux Projection Unification}. 
arXiv:YYYY.XXXXX [gr-qc].

\bibitem{Yilmaz2026d}
Yılmaz, K. (2026). 
\textit{Hyper-Flux Projection Model: Numerical Implementation and Cosmological Implications}. 
arXiv:YYYY.XXXXX [astro-ph.CO].

\bibitem{Bekenstein1973}
Bekenstein, J. D. (1973). 
Black holes and entropy. 
\textit{Physical Review D}, 7(8), 2333–2346.

\bibitem{Hawking1976}
Hawking, S. W. (1976). 
Breakdown of predictability in gravitational collapse. 
\textit{Physical Review D}, 14(10), 2460–2473.

\bibitem{Maldacena1999}
Maldacena, J. (1999). 
The large N limit of superconformal field theories and supergravity. 
\textit{International Journal of Theoretical Physics}, 38(4), 1113–1133.

\bibitem{Maldacena2013}
Maldacena, J., \& Susskind, L. (2013). 
Cool horizons for entangled black holes. 
\textit{Fortschritte der Physik}, 61(9), 781–811.

\bibitem{Page1993}
Page, D. N. (1993). 
Information in black hole radiation. 
\textit{Physical Review Letters}, 71(23), 3743–3746.

\bibitem{Unruh1976}
Unruh, W. G. (1976). 
Notes on black-hole evaporation. 
\textit{Physical Review D}, 14(4), 870–892.

\bibitem{Jaynes1957}
Jaynes, E. T. (1957). 
Information theory and statistical mechanics. 
\textit{Physical Review}, 106(4), 620–630.

\end{thebibliography}

\end{document}
